\documentclass[11pt]{article}

\usepackage[utf8]{inputenc}
\usepackage{graphicx}
\usepackage{float}
\usepackage[margin=.5in, labelfont = bf]{caption}
\usepackage[export]{adjustbox}
\usepackage[margin=0.5in]{geometry}

\graphicspath{{./images/}}

\begin{document}
    \title{Diverging Sprint Analysis}
    \author{Robert Purcaru}
    \date{\today}
    \maketitle


   In groups, people with louder voices tend to monopolize discussion. Since there is no clear correlation between this attribute and quality ideas or arguments, it makes sense to try to suppress this affect in favor of a more open and equal forum. This analysis will therefore focus on the strategies my group and I used throughout this sprint and how my focus on participation affected the product. This is shown through an outlining of two separate diverging exercises we performed, the first of which proved less fruitful than the second, followed by a discussion of how my focus on communication and participation of my group members affected the quality of the activities. 

    After receiving  the design brief, my group and I immediately began developing our individual ideas, using the 4-3-3 divergence tool [1]. During this part of the activity, I was not entirely engaged in the divergence effort in part because I was unsure of how my group mates liked to work and how they would respond to both relevant and irrelevant ideas. While we may have said many things, it didn't seem like much was being heard; it was as though while someone else was talking, everyone else was focusing on the idea they wanted to present instead of the idea that was being presented, something for which I was guilty of. This lead to us developing the contents of Figure \ref{FirstDesigns}.
    
    \begin{figure}[H]
        \centering\includegraphics[width = 0.8\textwidth]{FirstDesigns.PNG}
        \caption{A collection of several of our initial design ideas. Each set was made by a different person with little input from the rest of the group.}
        \label{FirstDesigns}
    \end{figure}

    While the work presented in Figure \ref{FirstDesigns} was useful, there were many short comings to it, most prevalent among which was the lack of inspiration behind them. Only 1 or 2 of the ideas shown could be fleshed out into candidates, which seems like a very low yield for how much time we spent on this portion, especially when one looks closely at Figure \ref{FirstDesigns} to see what we did. While I can't be sure, I'd speculate that this was a result of both our poor use of the tool and the fact that this was the first time my group members and I met and worked with each other. It seems likely however that having only one pair of eyes on each idea contributing to these shortcomings.

    It wasn't until later that we revisited all of our designs and I had the chance to try to get myself and the quieter members more involved. Since our first time through one of the tools proved relatively unsuccessful, I resolved to try something better on our round of revisions and critiques. I speculated that the challenging assumptions tool would give me an opportunity to try to give and receive solid feedback from my group members more so than the 4-3-3 method, leading to the production of Figure \ref{assumptions}.

    \begin{figure}[H]
        \centering\includegraphics[width = 0.7\textwidth]{assumptions.PNG}
        \caption{The results of our challenging assumptions exercise. Note that it does not take much effort to produce what is shown for how many avenues are opened.}
        \label{assumptions}
    \end{figure}

    This proved much more successful and relevant to all of the designs we produced. Interestingly, all subsequent structured diverging tools we used did not prove as successful. I can't be sure, but I believe this is because the 'challenging assumptions' exercise does not require a lot of structuring ('meta-work') on our part; one can see in Figure \ref{assumptions} that using this tool was just a matter of recording our ideas, no table or drawings necessary, and working off of those. Since I was interested in this tool, I took it upon myself to encourage the rest of my group to contribute individually. This seemed to allow the quieter members of the group to present ideas without the threat of having to defend the idea right away, since I would write the idea down and move on quickly, leaving the critique for later. This proved helpful in both getting people's voices heard and understanding different views on the same idea. In future diverging exercise, I'd like to bring both the attitude I had in the challenging assumptions exercise along with the tool itself, as both in part seemed to produce strong results.


    \section{References}
    \begin{verbatim}
        [1] R. Vullings, “27 Creativity & Innovation Techniques Explained in One-Pagers.” 
        https://blog.ramonvullings.com/post/47981894559/27-creativity-innovation-tools-is-an-
        overview-of, 2014.
    \end{verbatim}
  



\end{document}