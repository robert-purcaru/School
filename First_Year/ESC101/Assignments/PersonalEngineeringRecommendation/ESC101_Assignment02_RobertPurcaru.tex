\documentclass[11pt]{article}
\usepackage{textcomp, gensymb}
\usepackage[letterpaper, margin=1in]{geometry}
\usepackage[english]{babel}
\usepackage[dvipsnames]{xcolor}
\usepackage{float}
\usepackage[labelfont=bf]{caption}
\usepackage{graphicx}
\usepackage[hyphens]{url}
\usepackage{hyperref}
\hypersetup{colorlinks=false,breaklinks=false}
\graphicspath{ {./images/} }


\begin{document}
    \title{Selecting a Chair for Everyday Desk Use}
    \author{Robert Purcaru}
    \date{\today}
    \maketitle

    \section{Introduction}
    Over the past six months, I, along with many other people, have been spending a drastically increased amount of time sitting at a desk at home.  Since this seems unlikely to change in the immediate future, I have decided to purchase a dedicated chair for my desk. Currently, I'm using a solid kitchen chair without arm supports or any ability to adjust the sitting position. As a result, I tend to develop pain in my legs and back after sitting for a long time. To address this, I will use this report to compare a variety of options and determine which is most suitable for my needs.

    \section{Stakeholders}
    As the sole user and financier of the chair, I will be the only stakeholder considered during this analysis. As a result, I will be writing this report in the first person to make the report more readable. I tend to spend between 6 and 10 hours a day sitting in at my desk, during weekdays, while the chair I use currently has no designed ergonomic features besides a seat and a solid backrest. As a result of this, I tend to either sit overly erect or slouched in my chair, which results in overcompensation by my lumbar spine or my cervical spine, which can lead to back pain over time \cite{black1996influence}. The combination of the variety of options available and the potential risk of choosing an ergonomically unsuitable chair therefore justifies the existence of this report. Generally, I need a chair which is affordable, of sturdy enough construction to last years, and which is suitable for working at a corner desk. Additionally, I would prefer a chair which is in someway breathable to prevent sweating and feels comfortable to sit in. Stylistically, I prefer a chair that is either all black or black with blue accents that I can lean into.

    \section{Objectives, Constraints, Directives and Metrics}
    This analysis will consider two types of factors: constraints and directives. A constraint must be met for a product to be considered a viable solution, while an directive serves to support the choice of one product over another, generally indicating a preference the stakeholder has. While the directives are listed in order of value to me, the constraints must all be met or improved upon to make one product more competitive over another.
    \begin{quote}
        \subsection{Constraints}
        \begin{enumerate}
            \item \textbf{Price:} For a product to be considered 'affordable' it must cost no more than \$300 to get to my door. Generally, a cheaper chair is preferable to a more expensive one.
            \item \textbf{Base:} For use at a corner desk, the product must include a five point base with some form of caster wheels and the ability to rotate 360$^\circ$ about its long axis independent of the wheels (swivel), as specified in \cite{seating-viewing-considerations}. In addition to being recommended for office use, this layout also allows me to switch between the two sides of a corner desk without standing up.
            \item \textbf{Adjustability and Ergonomics:} I must be able to adjust the height of the seat as well as the angle of the backrest while in the sitting position. Additionally, the backrest must be adjustable to a minimum of 135$^\circ$, with respect to the seat, as this is the angle that results in a neutral lumbar spine a seated position \cite{keegan1953alterations}. Being able to extend past 135$^\circ$ will not be considered preferable to extending to a limit of 135$^\circ$.
            \end{enumerate}

        \subsection{Directives}
        \begin{enumerate}
            \item \textbf{Backrest Material:} To address overheating, the product should have some sort of mesh or otherwise breathable material comprising the backrest; I tend to sweat profusely after long periods of time on nonbreathable materials. This is my most important directive because I intend to keep this chair for a long time, and I would prefer if the material did not uptake the scent of sweat as I use it, nor would I find it comfortable to constantly sweat against the seat of the chair.
            \item \textbf{Seat Depth:} When seated on a flat surface on a flat surface with the back of my knees against the edge of the surface, 22.25in of my leg are in contact with the seat, so the seat on the product should be within 1in of 19.25in to accommodate my height \cite{howe_2016}.I've ranked seat depth above and separate from chair size because, in my experience, seat depth is often generalized to fit both taller and shorter people, leading to taller people getting the shorter end of the stick. In researching chairs and sitting in a variety of desk chairs prior to writing this report, I have noticed that chairs with deeper seats are considerably more comfortable for me. This is not my most important directive, although it is close, because I have sat in chairs that are considerably smaller than this before and only noticed a significant discomfort after very long periods of sitting.
            \item \textbf{Chair Size:} The chair should be fit for a person that is at least 6'5" tall and 200lbs, as specified by the manufacturer, and/or have a seat height that is reaches between 15in and 22in (38-56cm) off the ground and have a back rest that comes at least 18in (45cm) above the seat \cite{seating-viewing-considerations}. This is ranked third because I have no particular experience in this matter, however it seems generally agreed upon that this is an important factor for ergonomics.
            \item \textbf{Backrest Style:} From personal experience, chairs that have some sort of winging or flare built into the sides of backrest, shown in Figure \ref{fig:Wings} are more comfortable for me. While this is a practical measure in chairs found in places like race cars, for my purposes it is only for comfort. 
            \item \textbf{Armrests:} If the chair should come with armrest, it would be easier to use at a corner desk if the arm rests were removable without damaging the chair.
            \item \textbf{Style:} The chair should be available in all black or black with blue accents. 
        \end{enumerate}
        \begin{figure}
            \begin{center}
                \includegraphics[width = 2in]{Winglets.PNG}
            \end{center}   
            \caption{Winged backrest on a gaming chair. Wings on both sides are circled in red.}
            \label{fig:Wings}              
        \end{figure}   
    \end{quote}

    \section{Alternatives}
    
    The five chairs considered in this report are: the GTRacing Chair with Footrest, the Von Racer Gaming chair, the Executive and Manager Chair, the Flash Furniture High Back, and the Speedster Ergonomic Mesh Gaming Chair, Figure \ref{fig:chairs}. Link to purchase these products can be found in the Products subsection of the Appendix. 

    \begin{figure}[H]
        \includegraphics[width=6in]{chairDiagram.PNG}
        \caption{Images of all five alternatives with corresponding name.}
        \label{fig:chairs}
    \end{figure}
    
    
    The Von Racer Gaming chair and the GTRacing chair are both racing style gaming chairs. Both of these follow the general design style of most other racing style gaming chairs, however the GTRacing chair has a few gimmicks included like deep reclining (155$^\circ$) and swiveling arm rests. However, I'm not particularly interested in either of these so I've included a simpler model of a gaming chair, the Von Racer Gaming chair, to see if the chair meets my criteria while justifying its greater price. 
    
    The Executive and Manager chair and the Flash Furniture High Back are more conventional office style chairs, making them much simpler (stylistically) than the racing gaming, which feature winged seats and backrests. The Flash Furniture chair features an aggressive flare on the backrest, which is preferable to the flat back of the Executive and Manager chair. The main difference between the chairs however is the price. The Flash Furniture High back is significantly more expensive than the Executive and Manager chair, so I've included both to see if the features of the Flash Furniture High Back can justify the steeper price. 
    
    Lastly, the Speeder Mesh Gaming Chair is a sort of hybrid between the racing style chairs and the office chair, while also being the cheapest available option. It features a similar shape to the racing chairs and a common material composition with the office chairs. It was the only real middle-ground I could find between the two, so I've included it to see if it takes enough of the positive features of each to justify its purchase.

    \section{Assessment of Alternatives}
    For the purposes of this analysis, I started by comparing each chair to an 'ideal' chair which is described by all of the constraints and directives. Along these lines, I have constructed Table 1 to show how each metric compares to an ideal chair that exactly suits all of my metrics. 

    \hfuzz=38pt
    \begin{table}[H]
        \begin{center}
            \begin{tabular}{| p{1in} || p{1in} | p{1in} | p{1in} | p{1in} | p{1in} |}   
            \hline
            
            \textbf{Product} & GTRacing Chair 
            
            with Footrest & Von 
            
            Racer Gaming Chair & Executive 
            
            and
            
            Manager & Flash 
            
            Furniture 
            
            High Back & Speedster Mesh 
            
            Gaming Chair 
            \\
            \hline 

            \textbf{Price} 
                & \textcolor{green}{\$244} 
                & \textcolor{green}{\$200} 
                & \textcolor{green}{\$190}
                & \textcolor{green}{\$280}
                & \textcolor{green}{\$182}            
            \\
            \hline

            \textbf{Base} 
                & \textcolor{green}{5 point base}

                \textcolor{green}{360$^\circ$ swivel}
                & \textcolor{green}{5 point base}

                \textcolor{green}{360$^\circ$ swivel}      
                & \textcolor{green}{5 point base}

                \textcolor{green}{360$^\circ$ swivel}         
                & \textcolor{green}{5 point base}

                \textcolor{green}{360$^\circ$ swivel}           
                & \textcolor{green}{5 point base}

                \textcolor{green}{360$^\circ$ swivel}    
            \\
            \hline

            \textbf{Adjustability}
        
            and

            \textbf{Ergonomics} 
            &  \textcolor{green}{Adjustable height lever}
            
            \textcolor{green}{155$^\circ$ recline}
            & \textcolor{green}{Adjustable height lever}
            
            \textcolor{green}{135$^\circ$ recline}
            & \textcolor{green}{Adjustable height lever}
            
            \textcolor{green}{135$^\circ$ recline}
            & \textcolor{green}{Adjustable height lever,}

            \textcolor{green}{135$^\circ$ recline}
            & \textcolor{green}{Adjustable height lever}
            
            \textcolor{green}{135$^\circ$ recline}
            \\
            \hline

            Backrest 
            
            Material 
            &\textcolor{red}{Polyurethane Leather}
            &\textcolor{red}{Polyurethane Leather}
            &\textcolor{green}{Mesh}
            &\textcolor{green}{Mesh}
            &\textcolor{green}{Mesh}
            \\
            \hline

            Seat Depth
            &\textcolor{green}{19.68in}
            &\textcolor{green}{18.9in}
            &\textcolor{red}{18in}
            &\textcolor{red}{17.5}
            &\textcolor{red}{20.5}
            \\
            \hline

            Chair Size 
            &Recommended for up to  \textcolor{green}{6'5"} and \textcolor{green}{250lbs}
            &Recommended for up to  \textcolor{green}{6'5"} and \textcolor{green}{250lbs}
            &Seat adjusts \textcolor{green}{between 17.8in and 21.7in} and backrest sits \textcolor{green}{28"} above seat
            &Seat adjusts \textcolor{green}{between 17.5in and 21.8in} and backrest sits \textcolor{green}{21.8"} above seat
            &Seat adjusts \textcolor{green}{between 19in and 22in} and backrest sits \textcolor{green}{31"} above seat
            \\
            \hline

            Backrest Style
            &\textcolor{green}{Winged}
            &\textcolor{green}{Winged}
            &\textcolor{red}{Flat}
            &\textcolor{green}{Flared}
            &\textcolor{red}{Small Wings}
            \\
            \hline

            Armrest 
            &\textcolor{red}{Not Removable}
            &\textcolor{green}{Removable}
            &\textcolor{green}{Removable}
            &\textcolor{red}{Not removable}
            &\textcolor{green}{Removable}
            \\
            \hline

            Style 
            &\textcolor{green}{Balck and Blue}
            &\textcolor{green}{Balck and Blue}
            &\textcolor{green}{Black and Blue}
            &\textcolor{green}{Balck}
            &\textcolor{red}{Red and Black}
            \\
            \hline
            \end{tabular}
        \end{center}
        \caption{Comparison of all five alternatives against 'ideal chair'. Directives results are shown in green and red; where the alternative matched or fell within the specified range of acceptability for each metric is shown in green, and where it did not is shown in red. Each respective directive has a brief description of how it meets or doesn't meet the requirement specified.}
        \label{fig:table1}
    \end{table}

    Table 1 shows us that all the chairs meet the chair size specification. This could because most chairs are designed with these metrics in mind - metrics were taken from a book on ergonomics \cite{seating-viewing-considerations} - or that the metric is not specified well enough for me in particular.

    From Table 1, we see that between the gaming chairs, the Von Racer Gaming chair is more suited than the GTRacing chair because it is both \$44 less expensive and its armrests are removable. We also notice that the speedster Mesh Gaming chair is the only one that doesn't meet three out of the six directives. Therefore, we will discount the GTRacing chair on the grounds that it is an inferior racing chair compared to the Von Racer Gaming chair and the Speedster Mesh Gaming chair on the grounds that it misses the the most directives.

    For the next part of the analysis, we consider the office chairs. Both office chairs fall short in their seat depth, however the Flash Furniture falls a half inch shorter than the Executive and Manager. While the Flash Furniture does not have removable arm rests, it does have a flared backrest. However, this alone is not enough to compensate for the \$90 difference in price. We will therefore discount the Flash Furniture on the grounds that the grounds that the Executive and Manager offers most of the same features for a significantly reduced price.

    Finally, we are left with the Von Racer and the Executive and Manager. As stated in the Directives subsection, the difference in importance between backrest and seat depth is small; I can deal with not having either but would strongly prefer not to. However, notice that the Executive and Manager is only 0.25" shorter than the recommended minimum of 18.25". This amount, while slightly shorter than the recommended minimum, is still very close (The '2-4in from the back of the knee' provided by \cite{howe_2016} seems more like a guideline than an absolute limit). That in mind, the only other short coming of the Executive and Manager chair is its lack of lateral support on the backrest. The Von Racer on the other hand has no mitigating factor to support its material material; in my experience, polyurethane leather is not very breathable and can lead to sweating after prolonged periods of time. Therefore, the most suitable chair for my purposes is the Executive and Manager office chair.

    \section{Conclusion}
    This report recommends the Executive and Manager office chair as the best option for a desk chair suited for my needs. It is worth noting however that a racing chair like the Von Racer Gaming chair would also be a suitable alternative for someone who is less interested in seat material than I am.

    \newpage
    \section{Appendix}
    \subsection{Source Extracts}
    \cite{black1996influence} Poor posture and, in particular, poor sitting posture, is considered to be a major contributing factor in the development and perpetuation of back and neck pain. [...] The relation between lumbar and cervical postures is probably more complex than is generally acknowledged clinically. As the position of the lumbar spine moves toward extension, the cervical spine tends to move toward flexion and vice versa. However, the compensation in the cervical spine may occur at the upper cervical region, lower cervical region, or both. The inclination of the cervical base appears to be an important component in head and neck posture. 
    
    \cite{seating-viewing-considerations} The preferred chair type is a swivel chair on a five-point
    base, with a rounded front edge on the seat, easily height adjustable by the operator. [...]
    Seat height should be adjustable by the
    user within the recommended range of 15–22 in. (38-56 cm). [...]
    The top of the backrest should be at least 18 in. (45cm) above the compressed seat height. 
    

    \cite{keegan1953alterations} The normal curve of the lumbar spine in adult man is determined by maintenance of the trunk-thigh and the knee angles at approximately 135 degrees. Alteration of this normal lumbar curve, either an increase in standing erect or a decrease in sitting or stooping, is caused largely by the limited length and consequent pull of the trunk-thigh muscles of the opposite side. The most important postural factor in the causation of low-back pain in sitting is decrease of the trunk-thigh angle and consequent flattening of the lumbar curve.

    \cite{howe_2016} The seat should be deep enough so that the user can sit with his or her back against the backrest of the chair while leaving about 2"-4" between the back of the knees and the seat of the chair.

    \newpage
    \subsection{Products}
    \textbf{GTRacing Chair with Footrest:} \url{https://www.amazon.ca/dp/B087CQ214R/ref=twister_B089CS5V89?_encoding=UTF8&psc=1}
    \begin{center}
    \includegraphics[height = 3in]{GTRacingChair.PNG}
    \end{center}


    \textbf{Von Racer Gaming Chair:} \url{https://www.amazon.ca/VON-RACER-Massage-Gaming-Chair/dp/B0841YQ5YV/ref=pd_di_sccai_2/146-4468743-8974009?_encoding=UTF8&pd_rd_i=B0841YCFHR&pd_rd_r=d50b9957-8a68-4870-a66d-5b3562a2e16b&pd_rd_w=4lcuQ&pd_rd_wg=a13Q0&pf_rd_p=a5f9e780-487d-4881-a2b8-b9a00c4b0707&pf_rd_r=XYT5E2V9ZKMQWQ8NMZRH&refRID=XYT5E2V9ZKMQWQ8NMZRH&th=1}
    \begin{center}
    \includegraphics[height = 3in]{VonRacer.PNG}
    \end{center}
    
    \newpage
    \textbf{Executive and Manager Chair:} \url{https://www.amazon.ca/Executive-Adjustable-Breathable-Function-Blue-163/dp/B08CDMZTB4/ref=sr_1_43?dchild=1&keywords=mesh\%2Bdesk\%2Bchair&qid=1601597179&refinements=p_85\%3A5690392011&rnid=5690384011&rps=1&sr=8-43&th=1}
    \begin{center}
    \includegraphics[height = 3in]{Executive and Manager.PNG}
    \end{center}
    
    
    \textbf{Flash Furniture High Back:} \url{https://www.amazon.ca/dp/B00L48I838/ref=emc_b_5_t?th=1}
    \begin{center}
    \includegraphics[height = 3in]{FlashFurniture.PNG}
    \end{center}

    \newpage
    \textbf{Speedster Mesh Gaming Chair:} \url{https://www.lexmod.com/office/gaming-chairs/speedster-mesh-gaming-computer-chair-in-black-red/}
    \begin{center}
    \includegraphics[height = 3in]{Speedster.PNG}
    \end{center}

    \bibliographystyle{ieeetr}
    \bibliography{References}   
\end{document}