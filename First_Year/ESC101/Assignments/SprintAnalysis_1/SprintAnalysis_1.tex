\documentclass[11pt]{article}

\usepackage[utf8]{inputenc}
\usepackage{graphicx}
\usepackage{float}
\usepackage[margin=.5in, labelfont = bf]{caption}
\usepackage[export]{adjustbox}
\usepackage[margin=0.5in]{geometry}

\graphicspath{{./images/}}

\begin{document}
    \title{Framing Sprint Analysis}
    \author{Robert Purcaru}
    \date{\today}
    \maketitle

    In the past, whenever I tried to address an engineering problem, my approach was never particularly structured. As a result, I rarely thought very explicitly about some aspects of the solutions I was coming up with. Having completed the framing activity, I realize now the foremost element I failed to consider in the past was the stakeholders. To help cement the understanding I've developed in completing this assignment, this reflection will outline and explain the flawed approach my group and I took initially to developing a stakeholders section, the steps I took to try to correct the problem and understand why it occurred before closing with a brief 'lessons learned' for this activity to summarize my experience with this assignment.   

    Having decided upon what the focus of our design brief would be - addressing the planning fallacy in students - the first thing we did was develop a stakeholders section to set a foundation for the rest of our report. Initially, our group's understanding was that including more stakeholders would be better, because this would allow us to consider a wider variety of needs and would therefore lead to the development of a product that was more robust as a result. To illustrate this, I've included a screenshot of an early draft of our stakeholders section in Figure \ref{allStakeholders}. 
    \\
    
    \begin{figure}[H]
        \centering\includegraphics[width = 0.95\textwidth]{AllStakeholders.PNG}
        \caption{One of our early drafts of the stakeholders section we developed. Note that the contents of the section are not as important for this point, just that we have a broad range of stakeholders with a broad variety of goals.}
        \label{allStakeholders}
    \end{figure}

    Once we started developing our objectives, we began to see why the failures in this approach. First we considered areas where stakeholder interests differed and worked out which stakeholder would take priority in these cases. In doing so, we found that not only was our list not remotely robust enough to support objectives being formed upon it, it was also spread so thin that pairs of stakeholders were holding the exact same objectives in every case, causing us to question whether or not it was worth differentiating between them or excluding one outright. 
    
    With my group moving on to focus on other aspects of the assignment, I stayed back to develop the stakeholders section so that it might better help a reader understand our opportunity, instead of just ticking off boxes on a rubric. Generally, when I'm stick in a position like this I like to consult documentation available to make sure I understand what is intended or required from a specific task. Consulting the lecture on stakeholders provided by Professor Sheridan, I found a slide that contained two pages from a book on reference engineering to be particularly helpful \cite{book1}\footnote{I did not include a figure containing the two pages I'm referring to because it would have forced me over 2 pages for it to be legible.}. The section presented a model which outlined what could reasonably be considered a stakeholder and why, implying a degree of exclusivity for other matters. The part I found of particular use was in the introduction to the section where it is described how one could think of stake holders in terms of degrees of separation from the opportunity, something I had not considered while we were initially working on our stakeholders section. Using this model, I went back to our stakeholders section and reconsidered which were reasonable to include. With that, I developed the final stakeholder section in Figure \ref{finalSection}.
    
    \begin{figure}[H]
        \centering\includegraphics{FinalStakholders.PNG}
        \caption{Stakeholder Section from file our group submitted. Again, the contents of the figure are not critical, just that it is a much more compact section.}
        \label{finalSection}
    \end{figure}
    
    It's clear that the final draft is shorter than the first, however I think that, despite this, the information it provides is richer than the early draft. I think I accomplished this by focusing on information that was more clearly relevant to what we wanted to express, instead of presenting as much information as I could and hoping the point would get across through sheer volume. Questioning what information I include in my writing is something I'd like to carry over not only into future Praxis assignments but also whatever other writing I do in the future.


    %use requirements model, Perceive interpret act model (PIA)
    %CHOOSE DFX and Values
        %What does DFX now mean to you
        %what have you gained about the process    
    %figure come from our work
    %talk about fish bowl 


\bibliographystyle{ieeetr}
\bibliography{References.bib}

\end{document}