\documentclass[11pt]{article}
\usepackage{navigator}
\usepackage{mathtools} 
\usepackage{graphicx}
\usepackage{subcaption}
\usepackage[a4paper, total={6in, 9.5in}]{geometry}
\usepackage{float}
\usepackage{url}
\usepackage[margin=.5in, labelfont = bf]{caption}
\usepackage[normalem]{ulem}
\usepackage{xcolor,cancel}
\usepackage[raggedright]{titlesec}


\graphicspath{ {./images/} }

\begin{document}
    \title{Lab 2: Q Factor}
        \author{Robert Purcaru}
        \date{\today}
    \maketitle

    \tableofcontents
    
    \section{Author's Note}
        \subsection{General}
        This report features information pertaining to labs 2 through 4. Modifications and additions to existing sections are shown as \textcolor{red}{red text} embedded into the existing section, while new sections have the section header written in red while the body is black. Note that small or stylistic changes - like corrections in grammar or minor clarifications - are left in black. New Figures are shown with red captions.

        % Figures \ref{fig:QuadraticPowerSeries} and \ref{fig:CubicPowerSeries} show the graphs of quadratic and cubic power series. The captions are very similar just because the graphs depict very similar things, I included both to illustrate that continuing down the power series doesn't bring us closer to a non-zero C. A justification is included in section \ref{PeriodVsAmp}

        % \subsection{\textcolor{red}{Note to Nathan}}\label{note}
        % In my initial pendulum design, it proved impossible to capture a 90 degree range of motion because the camera was looking down at the pendulum (this was the only arrangement that worked as a result of the narrow space I have to work with). In an attempt to rectify this, I tried to manually find the horizontal distance the pendulum travelled using a ruler fixed against the backdrop and a camera that I moved as the amplitude dampened, recording the position every 5 oscillations. However, since I only have one ruler long enough to both measure the distance and be visible on camera, the affects of parallax became increasingly significant as the distance between the peak of the swing and the ruler increased. Since the ruler is about 6cm wide, and the vertical displacement of the mass is more than 25 times greater than that, the measurement uncertainty became more significant than the statistical uncertainty reported by the regression, making the results statistically zero. These results were excluded because the calculations to convert measured horizontal displacement needed to be tuned in a way that I could not justify with integrity to get the expected result.  My results will therefore only include data obtained in the original trial.
        
        % If this were to be attempted again, the measurement of the angle could be performed by placing the ruler near the pivot and measuring the points where the string can be seen overlapping with the ruler on the top and bottom of the ruler, followed by the use of trigonometry (given that the distance between the top and bottom of the ruler is a known quantity). This could potentially the measurement uncertainty down enough to allow us to consider our result statistically non-zero.

    \section{Experimental Design}\label{sec:design}
        The pendulum used in this experiment was constructed by fixing an adjustable G clamp to the top of a doorframe and suspending the mass from the throat of the clamp using a length of fishing line tied in fixed loop (bowline) at one end and attached to the weight at the other (Figure \ref{fig:setup}). This ensured a large range along which the length of the string could be adjusted and a wide field of view for the camera to capture. To prevent damage to the doorframe and allow for an increased clamping force, a pair of books were used between the clamp and the door frame. The clamp was tightened as much as was possible by hand to minimize any unwanted movement in the setup. The string length can be varied by using a shorter or longer piece of fishing line attached to the mass and the clamp in the same way. 

        \begin{figure}[H]
            \centering\includegraphics[width=.7\linewidth]{setup.jpg}
            \caption{Experimental setup used, excluding tripod and recording device. Left half of setup is obscured by wall.}
            \label{fig:setup}
        \end{figure}
        
        The mass used was composed of four 3/8-inch x 2-1/2-inch steel carriage bolts attached to a pair of 3D-Printed plates, one of which received an additional 3/16-inch x 1-1/2-inch steel eyebolt used to attach the mass to the end of the string. This design was chosen because it allowed for the location the string is mounted to to be varied continuously along the axis of the eyebolt, meaning it can be adjusted so that the string is closer to the center of mass for any number of carriage bolts. The mass can be adjusted by either removing or adding carriage bolts in pairs of two, giving four different configurations of the mass (Figure \ref{fig:masses}). Four carriage bolts were used for this experiment, despite the capacity for eight, because only four were available at the time the experiment was conducted. If more variation in the mass is required, larger plates can be printed to accept more carriage bolts or additional weights can be fixed to the part of the eyebolt the extends beyond the lower plate. The fishing line used was chosen because of its high tensile strength to weight ratio. 

        \begin{figure}[H]
            \centering\includegraphics[width = 0.75\linewidth]{allMasses.PNG}
            \caption{All combinations of bolts used to vary mass available with plates that were used for this experiment. The mass with four carriage bolts (second from the right) was used in this experiment.}
            \label{fig:masses}
        \end{figure}

        Before the mass was attached to the string, the position of the eyebolt was adjusted until the mass could be balanced side ways on the side of a pencil. However, once the mass was attached to the string, this configuration proved very unstable; the mass would rock about the point the string was attached to violently and irregularly. To address this, the eyebolt was raised by about $6 \pm 0.5$ threads on the eyebolt, corresponding to $0.50 \pm 0.02$ inches, or $1.27 \pm 0.05$cm. The string was then attached before its length was measured several minutes later to allow the pendulum to stabilize and to account for any strain in the string lengthening it. The length of the string was measured using a soft tape measure and found to be $184.5 \pm 0.1$cm between the top of the clamp and the nearest end of the interior of the loop of the eyebolt. However, since the eyebolt was raised by $1.27 \pm 0.05$cm, the distance between the center of mass and the pivot in the neutral position can be calculated as $184.5 \pm 0.1 + \frac{1.27 \pm 0.05}{2} = 185.1 \pm 0.1$cm, assuming the change in position of the center of mass is negligible from moving the eyebolt. \textcolor{red}{For experiments where the length of the string was adjusted, the same measurement uncertainty of $\pm$ 0.1cm is used for the distance between the pivot and the center of mass.}

        The masses of the mass and string were measured using a kitchen scale, giving $269.8 \pm 0.1$ grams for the mass and $0.2 \pm 0.1$ grams for the mass of the string. \textcolor{red}{Subsequent which involve various masses were measured the same way and will thus use the same uncertainty of $\pm$0.1 grams.} Since the mass of the string is three orders of magnitude smaller than the mass of the mass, we can consider it negligible for the purposes of this experiment.
        It was also found that the grooves cut into the throat of the G clamp were sufficient to prevent excessive lateral movement of the string while the experiment was being conducted (Figure \ref{fig:clamp}).
        \begin{figure}[H]
            \centering\includegraphics[angle = 270, width=.45\linewidth]{clampView.jpg}
            \caption{View of string hanging between teeth on clamp.}
            \label{fig:clamp}
        \end{figure}

       Finally, a measuring stick was placed underneath the pendulum and aligned with the eyebolt so that the 0 centerline marker lined up with the neutral axis of the pendulum (Figure \ref{fig:zeroed}). The video was taken from a top down view at a resolution of 1080x1920 pixels, where the pendulum swings along the axis of the 1920 pixels.

        \begin{figure}[H]
            \centering\includegraphics[width = 0.4\linewidth]{zeroed.jpg}
            \caption{Side view of mass zeroed against ruler.}
            \label{fig:zeroed}
        \end{figure}

    
    \section{Data}
        To collect data, a video was recorded by mounting an iPhone X to a tripod and placing it close to the string approximately, 1.3 meters above the mass, giving a top down view of the pendulum. This video was then analyzed in python using the OpenCV library \cite{opencv_library}. The program used looks at every frame in the video for the specified color, approximates the shape formed by the color to a circle, and reports the pixel position of the center of the circle. (Figure \ref{fig:opencvView}).

        \begin{figure}[H]
            \centering\includegraphics[width = 1\textwidth]{combo.PNG}
            \caption{Screenshot of thresh (left) and contour (right) views provided by OpenCV during video. The thresh view shows which areas of the frame are within the specified colour threshold determined in advance by identifying the RGB values of the yellow mass. The contour shows the circular approximation made by OpenCV in green and the center of the circle in blue. The position in pixels from the top of the frame is recorded every time the mass can be identified.}
            \label{fig:opencvView}
        \end{figure}


        \begin{figure}[H]
            \centering\includegraphics[width = 0.5\textwidth]{rawData.png}
            \caption{Raw data as mapped by OpenCV at specified frame. Video was recorded at 29.97 frames per second. Three distinct outliers can be seen around 520 seconds, 730 seconds, and 790 seconds. These occur because a foot entered the frame at those times, causing a tracking error in OpenCV. Error bars are excluded from this graph because data from OpenCV is interpreted with uncertainties in Figure \ref{fig:correctedGraph}; this graph only depicts information developed by OpenCV.}
            \label{fig:rawData}
        \end{figure}

        \begin{figure}[H]
            \centering\includegraphics[width = 1\textwidth]{correctedGraph.PNG}
            \caption{Data collected at specified frame, correcting for outliers and tracking errors after approximately N cycles. The graph featuring 3.5 cycles (left) depicts vertical error bars marking an uncertainty of$\pm$0.02 radians (see \ref{SourcesOfError:AnalysisQFactor}) and horizontal error bars marking $\pm$0.02 seconds (half of a frame). Graphs for 10.5 and 352 cycles (center and right) are shown to better illustrate the decay of the amplitude of the pendulum. It is worth noting that during the first few cycles the mass goes out of frame at its maximum amplitude on one side. This can be seen in the apparent sharp edges in the early troughs of the graph depicting 10.5 cycle and the 'blunt' edge of the bottom left side of the graph depicting 352 cycles.}
            \label{fig:correctedGraph}
        \end{figure}

        \begin{figure}[H]
            \centering\includegraphics[width = 0.8\textwidth]{tauGraph.PNG}
            \caption{Graph depicting relative maximums at specified times as determined by Open CV (green) with overlaid regression (red) in the form $\theta(t) = \theta_oe^{-t/\tau}$, where $a = 0.198$ and $\tau = 1380$ (left). The graph depicting the residuals (blue) between the data and the regression (right) illustrates some significant discrepancy between the model and the measurement (a black line at 0 rad discrepancy is used to help illustrate this). This is further discussed in \ref{SourcesOfError:AnalysisQFactor}. Both graphs feature error bars (orange) every 5 data points, illustrating a measurement error of 0.02 radians in the amplitude. Note that there are some large gaps between error bars due to problems determining a local maximum for certain times as a result of measurement errors.}
            \label{fig:bestFit}
        \end{figure}

        \begin{figure}[H]
            \centering\includegraphics[width = \textwidth]{QuadraticPowerSeriesData.png}
            \centering\includegraphics[width = \textwidth]{QuadraticPowerSeriesResiduals.PNG}
            \caption{Graph depicting period of pendulum as a function of amplitude, shown in blue, modeled against regression of quadratic power series in the form $T = T_{o} + B\theta_{0} + C\theta_{o}^{2}$ \cite{labManual}, shown in in green, (top). Uncertainties, shown in orange, associated with the period (vertical axis) are the measurement uncertainty of 0.2 seconds and the uncertainty in the amplitude (horizontal axis) is the measurement error of 0.02 rad (see \ref{SourcesOfError:MeasurementErrors} for explanation of both). The line of best fit has parameters $T_o = 2.72$, $B = -0.02 $, and $C = 0.4$. The values were found using the optimize library in SciPy \cite{2020SciPy-NMeth}; a discussion of the uncertainties associated with these values can be found in section \ref{PeriodVsAmp}. The 'jump' in the graph between about -0.11 rad and 0.11 rad is a result of the recording being stopped before the pendulum reached rest. Otherwise, the data used was recorded in the same manor as described in section \ref{sec:design}. The second graph (bottom) shows the residuals between the curve and the data. It can be seen that all of the data points fall within the uncertainty, meaning we can consider the result to be statistically zero for the amplitudes used when modeled against a quadratic power series. A calibration line at 0 is provided in gray.}  
            \label{fig:QuadraticPowerSeries}        
        \end{figure}

        \begin{figure}[H]
            \centering\includegraphics[width = \textwidth]{CubicPowerSeriesData.png}
            \centering\includegraphics[width = \textwidth]{CubicPowerSeriesResiduals.png}
            \caption{Graph depicting period of pendulum as a function of amplitude, shown in blue, modeled against regression of cubic power series in the form $T_o = 2.72 \pm 0.02 sec$, $B = 0.0 \pm 0.2 \frac{s}{rad}$,  $C = 0.4 \pm 0.8 \frac{s}{rad^2}$, and $D = -2 \pm 5 \frac{s}{rad^3}$ \cite{labManual}, shown in in red, (top). Uncertainties, shown in orange, associated with the period (vertical axis) are the measurement uncertainty of 0.2 seconds and the uncertainty in the amplitude (horizontal axis) is the measurement error of 0.02 rad (see \ref{SourcesOfError:MeasurementErrors} for explanation of both). The line of best fit has parameters $T_o = 2.7$, $B = 0.04 $, $C = 0.4$, and $D = -2$. The values were found using the optimize library in SciPy \cite{2020SciPy-NMeth}; a discussion of the uncertainties associated with these values can be found in section \ref{PeriodVsAmp}. The 'jump' in the graph between about -0.11 rad and 0.11 rad is a result of the recording being stopped before the pendulum reached rest. The second graph (bottom) shows the residuals between the curve and the data. It can be seen that all of the data points fall within the uncertainty, meaning we can consider the result to be statistically zero for the amplitudes used when modeled against a cubic power series. A calibration line at 0 is provided in gray.}
            \label{fig:CubicPowerSeries}        
        \end{figure}

        \begin{figure}[H]
            \centering\includegraphics[width = \textwidth]{AllRawLengthTimes.PNG}
            \caption{\textcolor{red}{This figure contains a summary of the raw data collected for the half period from various specified string length and constant mass. All points were found by counting the number of frames between when the center of the mass stopped moving in one direction to when the mass stopped moving in the other direction - half of a complete oscillation - and converting that measurement to a time using the frame rate of the video (29.92 fps). A table showing the raw data in terms of frames along with the averages used in Figure \ref{fig:GraphedLengths} can be found in Appendix \ref{app:RawData}. The purple horizontal line in all of the graphs represents the average of the 10 points shown for that particular graph. The uncertainties shown represent the length of one frame; the software used to track the pendulum reported the mass as still (not moving) for about 3 frames. In these cases, the count was started on the second frame. The standard deviations of each data set are: $\sigma_a = 0.02s$, $\sigma_b = 0.02s$, $\sigma_c = 0.02s$, $\sigma_d = 0.01s$, $\sigma_e = 0.02s$, $\sigma_f = 0.02s$. These were calculated using equation \ref{eq:stdev}. Note that all are less than the measurement uncertainty of 0.03 seconds}}
            \label{fig:RawLengthData}
        \end{figure}

        \begin{figure}[H]
            \includegraphics[width = \textwidth]{PowerFitLengths.png}
            \includegraphics[width = \textwidth]{LogFitForLengths.png}
            \caption{\textcolor{red}{Graph of measured period as a function of string length. All uncertainties are the measurement uncertainty of 0.03 seconds (1 frame); this is greater than the statistical uncertainty of each point. The data was modeled against the function $T = k(L_o + L)^n$. Values of 2.1 $\pm$ 0.1 for $k$, -0.08 $\pm$ 0.09 for $L_o$, and 0.48 $\pm$ 0.04 for $n$ where obtained using methods described in Section \ref{sec:StringVsPeriod}. A graph with linear axes is shown on top and a graph with logarithmic axes is shown on the bottom.
            \label{fig:GraphedLengths}}}
        \end{figure}

        \begin{figure}[H]
            \includegraphics[width = \textwidth]{AllRawMassTimes.PNG}
            \caption{\textcolor{red}{This figure contains a summary of the raw data collected for the half period from various specified masses with a  constant string length. All points were found by counting the number of frames between when the center of the mass stopped moving in one direction to when the mass stopped moving in the other direction - half of a complete oscillation - and converting that measurement to a time using the frame rate of the video (29.92 fps). A table showing the raw data in terms of frames along with averages can be found in Appendix \ref{app:RawData}. The purple horizontal line in all of the graphs represents the average of the 10 points shown for that particular graph. The uncertainties shown represent the length of one frame; the software used to track the pendulum reported the mass as still (not moving) for about 3 frames. In these cases, the count was started on the second frame. The standard deviations of each data set are: $\sigma_a = 0.02s$, $\sigma_b = 0.02s$, $\sigma_c = 0.01s$, $\sigma_d = 0.02s$, $\sigma_e = 0.02s$, $\sigma_f = 0.02s$. These were calculated using equation \ref{eq:stdev}. Note that all are less than the measurement uncertainty of 0.03 seconds. The averages for each are: $\sigma_a = 43.7s$, $\sigma_b = 43.8s$, $\sigma_c = 43.9s$, $\sigma_d = 43.7s$, $\sigma_e = 34.6s$, $\sigma_f = 43.7s$}}
            \label{fig:RawMassData}
        \end{figure}

        \begin{figure}[H]
            \includegraphics[width = \textwidth]{PeriodVsMass.png}
            \includegraphics[width = \textwidth]{ResidualsPeriodVsMass.png}
            \caption{\textcolor{red}{This figure shows a summary of the data collected to analyze the relationship between period and mass. The figure on the top represents the averages of all 7 data sets with error bars representing an uncertainty of 0.03s, corresponding to 1 frame. The outlier at 430.7 grams is removed for reasons discussed in Section \ref{sec:PeriodVsMass}. The residuals between remaining points and their average are shown on the bottom. It can be clearly seen that each point is statistically consistent with the average.}}
            \label{fig:AnalyzedMassVsPeriod}
        \end{figure}

    \section{Sources of Error}        
        \subsection{Measurement Errors}\label{SourcesOfError:MeasurementErrors}
        The pendulum and its amplitude was supervised throughout the experiment. The exposed portion of the eyebolt was used as a reference to find the position of the pendulum against the ruler underneath the pendulum. The ruler was zeroed within 0.0625 inches (Figure \ref{fig:zeroed}) so that any error resulting from an offset as the ruler was read could be minimized. Still, the measurement error here is significant. Given that the ruler was being read while the weight was moving, an uncertainty of about $\pm$ 0.3 inches seems reasonable, as this gives 4 increments in both direction to account for parallax and reading error at the start of the experiment. As the experiment proceeded, and the pendulum slowed down, it became easier to read the ruler, thereby increasing the confidence of the measurement and decreasing the error to about $\pm$0.1 inches, or about 2 increments on the ruler in both directions. This corresponds to 6 pixels, and using Equation \ref{calc:thetaApprox}, we find that the uncertainty in the amplitude is 0.02 radians, or about 1\% of the maximum amplitude. This explanation is confirmed with the results obtained in equation \ref{eq:theta}
        
        \textcolor{red}{In order to gather data on period to test whether it is affected by mass and string length, the same OpenCV library was used to determine when the pendulum had traveled half of a cycle, justification for which can be found in section \ref{sec:StringVsPeriod}. Since the period could only be measured as an integer representing the frame on which the motion ceased, there is an inherant uncertainty of $\pm$1 frame in all of these measurements. As one would expect, all the datapoints were within 1 frame of eachother, indicating that the actual period was somewhere between the two values measured. Since the frame rate of the video is given as 29.92 frames per second, the measurement uncertainty for all of these quantities is $\pm$0.03s. This information is represented in Figures \ref{fig:RawStringLengths} and \ref{fig:RawMassData}, with the raw data available in Appendix \ref{app:RawData}. A discussion about the uncertainties as they pertain specifically to the determining the relationship between period and mass, and period and sting length can be found in the first part of Section \ref{sec:Lab4Main}.}

        \subsection{Analysis Errors}\label{SourcesOfError:AnalysisQFactor}
        The OpenCV library doesn't explicitly provide any information for uncertainty or accuracy of the values provided. However, from experience, OpenCV provides  the center of a relatively good circle to within about a pixel. Since the circle identified was broken by the bolts (Figure \ref{fig:opencvView}), the error is likely around 8 pixels at the start when the pendulum is moving at its fastest, and decreasing as it slows down, resulting from motion blur. Using Equation \ref{calc:thetaApprox}, we obtain a value of $\pm 0.02$ rad for our uncertainty in the reported amplitude of the pendulum. As the pendulum slows and motion blur become less significant, this value likely decreases, however we will use $\pm 0.02$ radians as it is the largest source of uncertainty we find. This is a significant amount and should be addressed in future experiments. The measurements can be made more accurate and more frequently if OpenCV can identify the mass more consistently. To do this, a cover that is an unbroken color should be made to obscure the nuts on the top of the mass. This will allow the OpenCV to develop a better approximation of the pendulum, thereby reducing the error. It would also be worth trying a view perpendicular to the axis of rotation ('in front' of the pendulum) as this would largely eliminate measurement errors due to parallax.

        For the purposes of measuring the relationship between period and amplitude, we must consider how the period is measured. The data collected by OpenCV often reported the mass as near stationary (the change in position between frames was less than 3px) for up to six frames at a time (0.2 seconds), during which the reported position would vary unpredictably around some average value. This happened most commonly during the later stages of the experiment when the pendulum was moving more slowly. In order to determine the amplitude, I took all of the points where the mass was reported as 'stationary' and averaged the position that was reported along with the corresponding time stamps. The period was then determined by taking the difference between neighboring time stamps. The error in the reported period is therefore $\pm 0.2$ seconds, corresponding to the greatest number of near stationary frames that were reported. The only way this could be improved is with a higher definition recording device that could give OpenCV more pixels to analyze and therefore better report the position of the pendulum.

    \section{Q Factor}
        \subsection{Counting Cycles}
            By watching the first cycle and later reviewing the video manually, it was found the initial amplitude was  $13.75 \pm 2\%$ inches, as read on the ruler bellow. This corresponds to about $905 \pm 2\%$ pixels of the video and $0.22 \pm 2\%$ rad. If we consider the parallax error in reading the ruler negligible, we find that the amplitude corresponding to $e^{-\pi}$ of 14.0 inches is about $0.605 \pm 2\%$ inches. This corresponds to about $41 \pm 2\%$ pixels. Counting in real time, I found that the amplitude reached 0.605 inches after 328 complete oscillations. Reading the raw data, we find that the amplitude reaches ~4\% of its original amplitude after $332 \pm 1\%$ cycles, corresponding to 15 minutes and 5 seconds of video time. The uncertainty was adapted from the raw data, where the position was reported by openCV to 10 decimal places on a linear pixel scale of 1 unit to 0.1 pixels. Taking the width of the frame (1920px), we find an uncertainty that rounds to 1\%. While these two values are very close and within the uncertainty of the other, the uncertainty of reading the ruler is greater than the uncertainty using OpenCV; the ruler used only has marks every 0.0625 inches, which must be judged to line up with the center of the weight in real time. The data generated by the program on the other hand has an estimated error of 8 pixels, which likely decreases as the pendulum speed decreases. We will therefore consider $332 \pm 1\%$ as the Q factor determined by counting the oscillations until ~4\% amplitude is reached. 
        
        \subsection{Determining $\tau$}
            To affirm our value of Q obtained by counting, we can consider the formula
            \begin{equation}
                Q = \pi \frac{\tau}{T}
                \label{eq:qTauRelation}
            \end{equation} 
            where $T$ is the period of one complete oscillation and $\tau$ is the time constant of decay, as provided by \cite{labManual}. To find the period, we can take the average of the differences of neighboring maximums to get 2.73s. To find the error in this measurement, we can take the standard deviation of the difference of consecutive maximums, giving us 0.07s. Therefore, the period of one oscillation is 2.73 $\pm$ 0.07 seconds.
            To find a value for $\tau$, we can use the formula
            \begin{equation}
                \theta(t) = \theta_oe^{-t/\tau}cos(2\pi\frac{t}{T}+\phi_o)
                \label{eq:theta}
            \end{equation}
            where $\theta(t)$ is the amplitude in radians as a function of time, $\theta_o$ is the initial amplitude in radians, and $\phi_o$ is the phase constant that accounts for the difference between when the analysis starts and the pendulum starts moving, as provided by \cite{labManual}. If we only consider the maxima of our data, $cos(2\pi\frac{t}{T}+\phi_o) = 1$, so we are left with 
            \begin{equation}
                \theta(t) = \theta_oe^{-t/\tau}
                \label{eq:shortTheta}
            \end{equation}
            The maximums calculated and the line of best fit approximated using Equation \ref{eq:shortTheta} are shown in Figure \ref{fig:bestFit}. The value for $\tau$ used in the approximation is 1380 $\pm$ 3\% seconds. This is provided by the square root of the first element in the covariance matrix give by the curve fit function in the SciPy library \cite{2020SciPy-NMeth}. However, it is clear from the graph that this approximation is very rough. Keeping that in mind, the value we calculate for using this method is 1590 $\pm$ 3\% (Equation \ref{calc:tauApprox} in \ref{app:Calc}). This value may be excessively large as a result as the relatively small original amplitude; about $0.218 \pm 3\%$ radians (about 12.5 degrees). A small amplitude like this leads to a decreased maximum speed and, assuming air resistance, given by the equation:
            \begin{equation}
                D = Cd\rho\frac{v^2A}{2}
            \end{equation}
            
            taken form \cite{nasa_2015}, is the largest slowing factor, the pendulum would be impacted less by air resistance at slower speeds ($v$). However, this may also be a result of an analysis error. Future experiments will attempt to reduce the uncertainty resulting from analysing the data using methods described in \ref{SourcesOfError:AnalysisQFactor}.

    \section{Relationship Between Period and Amplitude}\label{PeriodVsAmp}
        Using the SciPy optimize library \cite{2020SciPy-NMeth}, I performed a regression on my data using both quadratic and cubic power series, seen in figures \ref{fig:QuadraticPowerSeries} and \ref{fig:CubicPowerSeries}. The parameters and the respective uncertainties determined for the quadratic power series in the form 
        \begin{equation}
            T = T_{o} + B\theta_{0} + C\theta_{o}^{2}
        \end{equation}
        are: $T_o = 2.72 \pm 0.02 s$, $B = -0.02 \pm 0.04 \frac{s}{rad}$, and $C = 0.4 \pm 0.8 \frac{s}{rad^2}$. From this, we find that both B and C are consistent with zero. Since B is consistent zero, the pendulum is likely not significantly asymmetric at the amplitude range tested. Furthermore, since the value of C is statistically zero, we can say that for the amplitude range tested, amplitude and period are independent. Therefore, in future labs, this range will be used to measure the relationship between different variables in the setup. 
        
        Similarly, when I performed the cubic regression with the function 
        
        \begin{equation}
            T = T_{o} + B\theta_{0} + C\theta_{o}^{2} + D\theta_{o}^3
            \label{eq:PeriodVsAmp}
        \end{equation}
        
        The regression found $T_o = 2.72 \pm 0.02 sec$, $B = 0.0 \pm 0.2 \frac{s}{rad}$,  $C = 0.4 \pm 0.8 \frac{s}{rad^2}$, and $D = -2 \pm 5 \frac{s}{rad^3}$. Again, B and C are statistically zero, meaning the conclusion we developed using the quadratic equation is corroborated by the cubic regression.

        Both quadratic and cubic regressions produce a value of 2.72 $\pm$ 0.02 seconds for the period of the pendulum. However, the measurement uncertainty of 0.2 seconds is greater. Therefore the period of the pendulum in the range of tested amplitudes is 2.7 $\pm$ 0.2 seconds, or 2.7 seconds $\pm$ 7\%.

        The ranges shown in figures \ref{fig:QuadraticPowerSeries} and \ref{fig:CubicPowerSeries} were determined by the number of oscillations measured. While initially, I planned to include as many oscillations as were performed in my Q factor experiment, it became clear from the data collected during the Q lab that measuring the amplitude of smaller oscillations would be significantly more challenging compared to measuring the amplitude of larger oscillations. This was a result of the amount time it seemed the pendulum was still for for small values of $\theta$. A balance seemed to exist around 0.1 radians of amplitude where the time it the mass seemed to be still was less than or equal to about six frames (about 0.02 seconds).   

        I chose to include the cubic regression to see if continuing down the power series would cause the value of C to converge, potentially giving a statistically non-zero result after a certain number of iterations. However, C remained statistically zero through the cubic power series, leading me to conclude that, within the range of amplitudes I tested my pendulum, period is independent of amplitude. 
        
        If this were to be attempted again, the measurement of the angle could be performed by placing the ruler near the pivot and measuring the points where the string can be seen overlapping with the ruler on the top and bottom of the ruler, followed by the use of trigonometry (given that the distance between the top and bottom of the ruler is a known quantity). This could potentially the measurement uncertainty down enough to allow us to consider our result statistically non-zero.

    \section{\textcolor{red}{Determining Relationship Between String length and Period, and Mass and Period}}\label{sec:Lab4Main}
    To measure the period of each oscillation for the purposes of determining the affect of string length and mass on period, I measured several half cycles as the amplitude decreased bellow 12 degrees; this is within the range where period was determined to be independent of amplitude (refer to Section \ref{PeriodVsAmp}. Note specifically that for ranges smaller than 12 degrees, C in Equation \ref{eq:PeriodVsAmp} is consistent with 0). Several measurements were taken along a 0.5 cycle interval; 0.5 cycles were chosen since the Q factor was determined to be 332 $\pm$ 1\% and the smallest measurement uncertainty was associated with the longest string length; 194.0cm $\pm$ 0.2\%. To keep the error due to the pendulum slowing down to a minimum, a percent error of less than 0.2\% was used, corresponding to about 0.7 cycles. For convenience and ease of measurement, I took a measurement every 0.5 cycles for 10 cycles and then averaged the data to get an approximation for the period of each length of string. This information is summarized in Figures \ref{fig:RawLengthData} and \ref{fig:RawMassData} with the raw data available in Appendix \ref{app:RawData}. Note that in each case the measurement uncertainty of 1 frame (0.03 seconds) was greater than the statistical uncertainty determined by one standard deviation in Equation \ref{eq:stdev}.
        
        \subsection{\textcolor{red}{Relationship Between String Length and Period}} \label{sec:StringVsPeriod}
        {
        Once the averages were obtained, a regression was performed using the SciPy optimize library \cite{2020SciPy-NMeth} against the function
            \begin{equation}
                T = k(L_o + L)^n
            \end{equation}
        where $T$ is the period, $k$ and $n$ are constants, $L_o$ is the error in effective length of the pendulum and $L$ represents the measured length of the pendlum. A graph of the function along with the data points is presented in Figure \ref{fig:GraphedLengths}. The regression gives values of $2.1 \pm 0.1 \frac{s}{m}$ for $k$, $-0.08 \pm 0.09 m$ for $L_o$, and $0.48 \pm 0.04$ for $n$. The values for $k$ and $n$ are consistent with the model
            \begin{equation}
                T \approx 2\sqrt{L}
            \end{equation} 
        Note that $L_o$ is consistant with 0, inidcating that the measurements take for the position of the center of mass were statistically consistent with the actual position of the center of mass. 
        
        The largest uncertainty in this part of the experiment was the measurement of time; an uncertainty of 1 frame over a range of 25 to 44 frames corresponds to an uncertainty of between 2 and 4\%, which is much greater than any other measurement or statistical uncertainty; the rest of which are on the order of a tenth of a percent.

        If this experiment were to be conducted again, the measurement uncertainty of $\pm$ 0.03 seconds could be reduced by taking a slow motion video on the same iPhone at 240 frames per second, corresponding to a measurement uncertainty of $\pm$ 0.004 seconds.
        }

        \subsection{\textcolor{red}{Relationship Between Mass and Period}}\label{sec:PeriodVsMass}
        For this portion of the experiment, various masses, specified in Appendix \ref{app:RawData} and shown in Figures \ref{fig:RawMassData} and \ref{fig:AnalyzedMassVsPeriod}, were swung on the pendulum to measure the affect that mass has on the period. Masses ranging between 118.3 grams and 508.9 grams were used because this populated a spread with a factor of about 4.3 separating the least from the greatest. If this experiment were to be conducted again, a greater range of masses could be employed; a lighter mass would not work with this setup (the minimum number of carriage bolts and nuts were used for the 118.3 gram trial), however more mass could be added by stacking concentric washers bellow the eyebolt in such a way that they would move freely. It is likely that if the mass were to increase substantially, the strain in the string would also increase, potentially causing an appreciable elongation of the string which would affect the period, as determined in Section \ref{sec:StringVsPeriod}. 

        If we consider Figure \ref{fig:RawMassData}, we see that almost all of the averages are within the measurement uncertainty of each other. The only exception is the trial where 430.7 grams were tested (f in Figure \ref{fig:RawMassData}). Upon closer inspection of the video, it can be seen that the mass was spinning significantly more about its axis compared to the other masses. On this basis, we will discount the data from this trial and only consider the remaining 6.

        From the other 6 data sets, we see no statistically significant relationship between mass and mass and period, as can be seen in Figure \ref{fig:AnalyzedMassVsPeriod}. While the model predicts that different masses would appreciably affect Q factor, which in turn affects the measurement of period, no such relationship can be observed from the data collected here. It is possible that the range of masses was too narrow to make this affect significant enough to measure (The largest and smallest masses used were 508.9 grams and 118.3 grams). Furthermore, it seems that the string did not elongate under the stress of the mass enough to cause an appreciable difference in the period between trials.

        The largest source of uncertainty in this experiment was again the uncertainty in the time measurement, corresponding to an uncertainty of upwards of 2\% for all the trials. As was the case described in section \ref{sec:StringVsPeriod}, this could be improved by taking a high frame rate video to reduce this uncertainty by close to a factor of about 8. 

    \section{Appendix}
        \subsection{Equations and Calculations}\label{app:Calc}
            Approximation for Q using $\tau$ obtained in Figure \ref{fig:bestFit}. Uncertainties were obtained from the covariance table that was obtained from the SciPy regression, and is propagated as the largest percent among the two.
            \begin{equation}
                Q = \pi\frac{2.73 \pm 0.07}{1380 \pm 35} = 1590 \pm 40
                \label{calc:tauApprox}
            \end{equation}
            \begin{equation}
                Q = \pi\frac{2.73 \pm 3\%}{1380 \pm 3\%} = 1590 \pm 3\% 
                \label{calc:tauApprox}
            \end{equation}
            \\
            Conversion equation used to determine angle from position P, expressed in pixels. The camera was mounted such that the lens was within $5.0 \pm 0.1cm$ of the string attached to the mass. If we consider this distance to be negligible compared to the length of the apparatus, we find that the angle can be expressed as the arcsin of the horizontal position of the pendulum divided by the length of string. This also assumes that the difference between the mass and the measuring device is negligible (it was measured to be $4.8 \pm 0.1cm$ at the maximum distance the camera could still find it's position.)
            \begin{equation}
                \theta(P) = \arcsin{\frac{\frac{2.54cm}{in}\frac{31.875 \pm 0.25in}{1920px} P}{185.1cm \pm 0.1cm}} 
                \label{calc:thetaApprox}
            \end{equation}
            \begin{equation}
                \theta(P) = \arcsin{\frac{\frac{2.54cm}{in}\frac{31.875 \pm 1\%}{1920px} P}{185.1 \pm 0.1\%}} 
            \end{equation}
            \begin{equation}
                \theta(P) = \arcsin{(0.000228 * P)} \pm 0.1\% 
            \end{equation}
        
        To calculate the standard deviation, we use:
        \begin{equation}
            \sigma = \sqrt{\frac{\Sigma|x_i - \mu|^{2}}{N}}
            \label{eq:stdev}
        \end{equation}
        where $x_i$ is an entry from the set, $\mu$ is the average of the set, and $N$ is the size of the set. This equation was obtained from \cite{khan_2020}.

        \subsection{Programs Used}
            The following is an excerpt of the program that tracks the mass. The program looks for the colors specified in the given frame, approximates the region the color is identified in with a circle and records the center of the identified circle.
            \begin{verbatim}
                def find_mass(frame, show_trackbars = False):
                    frame = cv2.GaussianBlur(frame, (7, 7), 0)
                    hsv = cv2.cvtColor(frame, cv2.COLOR_BGR2HSV)

                    hul, huh, sal, sah, val, vah = 23, 40, 126, 255, 127, 255 

                    if show_trackbars:
                        hul=cv2.getTrackbarPos(hl, wnd)
                        huh=cv2.getTrackbarPos(hh, wnd)
                        sal=cv2.getTrackbarPos(sl, wnd)
                        sah=cv2.getTrackbarPos(sh, wnd)
                        val=cv2.getTrackbarPos(vl, wnd)
                        vah=cv2.getTrackbarPos(vh, wnd)

                    HSVLOW = np.array([hul, sal, val])
                    HSVHIGH = np.array([huh, sah, vah])
                    thresh = cv2.inRange(hsv, HSVLOW, HSVHIGH)
                    floodfill = thresh.copy()
                    h, w = thresh.shape[:2]
                    mask = np.zeros((h+2, w+2), np.uint8)
                    cv2.floodFill(floodfill, mask, (0,0), 255)
                    floodfill_inv = cv2.bitwise_not(floodfill)
                    thresh_filled = thresh | floodfill_inv    
                    edged = cv2.Canny(thresh_filled, 30, 200)
                    all_contours, hierarchy = cv2.findContours(edged, 
                    cv2.RETR_TREE, cv2.CHAIN_APPROX_SIMPLE)

                    filtered_contours = []

                    for contour in all_contours:
                        area = cv2.contourArea(contour)
                        x, y, w, h = cv2.boundingRect(contour)
                        if abs(max(w, h)/min(w, h)) < 1.5:
                            if area > 100:
                                filtered_contours.append(contour) 
                                break

                    if len(filtered_contours) < 1:
                        return -1, -1, -1, thresh_filled

                    mass_contour = filtered_contours[0]
                    center, radius = cv2.minEnclosingCircle(mass_contour)
                    
                    return int(center[0]), int(center[1]), int(radius), 
                    thresh_filled
    \end{verbatim}

    \subsection{\textcolor{red}{Raw Data}}\label{app:RawData}

    \begin{figure}[H]
        \includegraphics[width = \textwidth]{RawStringLengths.PNG}
        \caption{\textcolor{red}
        {Number of frames counted for various string lengths. Each value represents the number of frames counted between when the mass stopped moving to the left (negative direction) and when the mass stopped moving to the right (positive direction). The uncertainties in each of the lengths are $\pm$ 0.1cm, as determined in Section \ref{sec:design}.}
        \label{fig:RawStringLengths}
        }   
    \end{figure}
    
    \begin{figure}[H]
        \includegraphics[width = \textwidth]{RawMasses.PNG}
        \caption{\textcolor{red}
        {Number of frames counted for various masses. Each value represents the number of frames counted between when the mass stopped moving to the left (negative direction) and when the mass stopped moving to the right (positive direction). The uncertainties for each of the weight measurements is $\pm$0.1 grams, as discussed in Section \ref{sec:design}.}
        \label{fig:RawMasses}
        }   
    \end{figure}


    \bibliography{References.bib}
    \bibliographystyle{ieeetr}


\end{document}